% Chapter 1

\chapter{Conclusion and future directons} % Main chapter title

\label{Chapter1} % For referencing the chapter elsewhere, use \ref{Chapter1}

%----------------------------------------------------------------------------------------

% Define some commands to keep the formatting separated from the content
\newcommand{\keyword}[1]{\textbf{#1}}
\newcommand{\tabhead}[1]{\textbf{#1}}
\newcommand{\code}[1]{\texttt{#1}}
\newcommand{\file}[1]{\texttt{\bfseries#1}}
\newcommand{\option}[1]{\texttt{\itshape#1}}

%----------------------------------------------------------------------------------------

\section{Conclusion}

In Conclusion there are quite a few differences between Lojban and Atomese even though they are both based on predicate Logic. Understanding these differences does not only help in translating between the 2 but is also helpful in learning how to best represent things in the very flexible Atomese Language, as we can leverage the great amount of effort already put into designing Lojban.

\section{Future Work}

Since Lojban can be considered an Ontology it could be of interesting to take a closer look at how well reasoning on this works as compared to other Ontologies like SUMO or Cyc. The hypotheses would be that because Lojban is actually used by a group of human Speakers that have continuously developed the language for ease of use, that the Lojban to Atomese output would lend itself especially well for reasoning, because it can represent conceptually simple things in a logically simple way. 
