% Chapter 1

\chapter{Relevant Linterature} % Main chapter title

\label{Chapter1} % For referencing the chapter elsewhere, use \ref{Chapter1}

%----------------------------------------------------------------------------------------

% Define some commands to keep the formatting separated from the content
\newcommand{\keyword}[1]{\textbf{#1}}
\newcommand{\tabhead}[1]{\textbf{#1}}
\newcommand{\code}[1]{\texttt{#1}}
\newcommand{\file}[1]{\texttt{\bfseries#1}}
\newcommand{\option}[1]{\texttt{\itshape#1}}

%----------------------------------------------------------------------------------------

\section{FrameNet}

\subsection{FrameSemantics}

Frame Semantics is a theory developed by Charles J. Fillmore for understanding linguistic meaning. In it a word activates a frame which is basically a predicate relating various Entities.

\subsection{FrameNet}

FrameNet is a lexical database containing over 1200 semantic frames and 13000 lexical units as well as 202000 example sentences.

\section{SUMO}



\section{CYC}
